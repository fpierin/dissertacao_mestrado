% 
% Versão 5: Sex Mar  9 18:05:40 BRT 2012
%
% Criação: Jesús P. Mena-Chalco
% Revisão: Fabio Kon e Paulo Feofiloff
%  
% Obs: Leia previamente o texto do arquivo README.txt

\documentclass[11pt,twoside,a4paper]{book}

% ---------------------------------------------------------------------------- %
% Pacotes 
\usepackage[T1]{fontenc}
\usepackage[brazil,portuguese]{babel}
\usepackage[utf8]{inputenc}
\usepackage{listingsutf8}
%\usepackage[acronym, toc]{glossaries}   % Glossários
%\usepackage[latin1]{inputenc}
\usepackage[pdftex]{graphicx}           % usamos arquivos pdf/png como figuras
\usepackage{setspace}                   % espaçamento flexível
\usepackage{indentfirst}                % indentação do primeiro parágrafo
\usepackage{makeidx}                    % índice remissivo
\usepackage[nottoc]{tocbibind}          % acrescentamos a bibliografia/indice/conteudo no Table of Contents
\usepackage{courier}                    % usa o Adobe Courier no lugar de Computer Modern Typewriter
\usepackage{type1cm}                    % fontes realmente escaláveis
\usepackage{listings}                   % para formatar código-fonte (ex. em Java)
\usepackage{titletoc}
%\usepackage[bf,small,compact]{titlesec} % cabeçalhos dos títulos: menores e compactos
\usepackage[fixlanguage]{babelbib}
\usepackage[font=small,format=plain,labelfont=bf,up,textfont=it,up]{caption, subcaption}
\usepackage[usenames,svgnames,dvipsnames]{xcolor}
\usepackage[a4paper,top=2.54cm,bottom=2.0cm,left=2.0cm,right=2.54cm]{geometry} % margens
%\usepackage[pdftex,plainpages=false,pdfpagelabels,pagebackref,colorlinks=true,citecolor=black,linkcolor=black,urlcolor=black,filecolor=black,bookmarksopen=true]{hyperref} % links em preto
\usepackage[pdftex,plainpages=false,pdfpagelabels,pagebackref,colorlinks=true,citecolor=DarkGreen,linkcolor=NavyBlue,urlcolor=DarkRed,filecolor=green,bookmarksopen=true]{hyperref} % links coloridos
\usepackage[all]{hypcap}                    % soluciona o problema com o hyperref e capitulos
\usepackage[round,sort,nonamebreak]{natbib} % citação bibliográfica textual(plainnat-ime.bst)
\fontsize{60}{62}\usefont{OT1}{cmr}{m}{n}{\selectfont}

% ---------------------------------------------------------------------------- %
% Cabeçalhos similares ao TAOCP de Donald E. Knuth
\usepackage{fancyhdr}
\pagestyle{fancy}
\fancyhf{}
\renewcommand{\chaptermark}[1]{\markboth{\MakeUppercase{#1}}{}}
\renewcommand{\sectionmark}[1]{\markright{\MakeUppercase{#1}}{}}
\renewcommand{\headrulewidth}{0pt}

% ---------------------------------------------------------------------------- %
\graphicspath{{./figuras/}}             % caminho das figuras (recomendável)
\frenchspacing                          % arruma o espaço: id est (i.e.) e exempli gratia (e.g.) 
\urlstyle{same}                         % URL com o mesmo estilo do texto e não mono-spaced
\makeindex                              % para o índice remissivo
\raggedbottom                           % para não permitir espaços extra no texto
\fontsize{60}{62}\usefont{OT1}{cmr}{m}{n}{\selectfont}
\cleardoublepage
\normalsize

% ---------------------------------------------------------------------------- %
% Opções de listing usados para o código fonte
% Ref: http://en.wikibooks.org/wiki/LaTeX/Packages/Listings
\definecolor{codegreen}{rgb}{0,0.6,0}
\definecolor{codegray}{rgb}{0.5,0.5,0.5}
\definecolor{codepurple}{rgb}{0.58,0,0.82}
\definecolor{backcolour}{rgb}{0.99,0.99,0.99}

\lstset{ %
language=Java,                  % choose the language of the code
extendedchars=false,
inputencoding=utf8,
basicstyle=\footnotesize,       % the size of the fonts that are used for the code
numbers=left,                   % where to put the line-numbers
numberstyle=\footnotesize,      % the size of the fonts that are used for the line-numbers
stepnumber=1,                   % the step between two line-numbers. If it's 1 each line will be numbered
numbersep=5pt,                  % how far the line-numbers are from the code
showspaces=false,               % show spaces adding particular underscores
showstringspaces=false,         % underline spaces within strings
showtabs=false,                 % show tabs within strings adding particular underscores
frame=trbl,	                % adds a frame around the code
frameround=fttt,
framerule=0.6pt,
tabsize=2,	                    % sets default tabsize to 2 spaces
captionpos=b,                   % sets the caption-position to bottom
breaklines=true,                % sets automatic line breaking
breakatwhitespace=false,        % sets if automatic breaks should only happen at whitespace
escapeinside={\%*}{*)},         % if you want to add a comment within your code
rulecolor=\color[rgb]{0.8,0.8,0.8},
backgroundcolor=\color{backcolour},   
commentstyle=\color{codegreen},
keywordstyle=\color{magenta},
numberstyle=\tiny\color{codegray},
stringstyle=\color{codepurple},
xleftmargin=10pt,
xrightmargin=10pt,
framexleftmargin=10pt,
framexrightmargin=10pt
}

% ---------------------------------------------------------------------------- %
% Corpo do texto
\begin{document}
\frontmatter 
% cabeçalho para as páginas das sessões anteriores ao capítulo 1 (frontmatter)
\fancyhead[RO]{{\footnotesize\rightmark}\hspace{2em}\thepage}
\setcounter{tocdepth}{2}
\fancyhead[LE]{\thepage\hspace{2em}\footnotesize{\leftmark}}
\fancyhead[RE,LO]{}
\fancyhead[RO]{{\footnotesize\rightmark}\hspace{2em}\thepage}

\onehalfspacing  % espa

% ---------------------------------------------------------------------------- %
% CAPA
% Nota: O título para as dissertações/teses do IME-USP devem caber em um 
% orifício de 10,7cm de largura x 6,0cm de altura que há na capa fornecida pela SPG.
% Integração de repositórios de eventos na Web por meio de mapeamento semântico
\thispagestyle{empty}
\begin{center}
    \vspace*{2.3cm}
    \textbf{\Large{IntegraWeb: uma proposta de arquitetura baseada em
mapeamentos semânticos e técnicas de mineração de dados}}\\
    
    \vspace*{1.2cm}
    \Large{Felipe Lombardi Pierin}
    
    \vskip 2cm
    \textsc{
    Dissertação apresentada\\[-0.25cm] 
    ao\\[-0.25cm]
    Instituto de Matemática e Estatástica\\[-0.25cm]
    da\\[-0.25cm]
    Universidade de São Paulo\\[-0.25cm]
    para\\[-0.25cm]
    obtenção do título\\[-0.25cm]
    de\\[-0.25cm]
    Mestre em Ciências}
    
    \vskip 1.5cm
    Programa: Pós-Graduação em Ciência da Computação\\
    Orientador: Prof. Dr. Jaime Simão Sichman

   	\vskip 1cm
    \normalsize{}
    
    \vskip 0.5cm
    \normalsize{São Paulo, Janeiro de 2018}
\end{center}

% ---------------------------------------------------------------------------- %
% Página de rosto (SÓ PARA A VERSÃO DEPOSITADA - ANTES DA DEFESA)
% Resolução CoPGr 5890 (20/12/2010)
%
% IMPORTANTE:
%   Coloque um '%' em todas as linhas
%   desta página antes de compilar a versão
%   final, corrigida, do trabalho
%
%

%\newpage
%\thispagestyle{empty}
%    \begin{center}
%        \vspace*{2.3 cm}
%        \textbf{\Large{
%        IntegraWeb: uma proposta de arquitetura baseada em
%mapeamentos semânticos e técnicas de mineração de dados}}\\
%        \vspace*{2 cm}
%    \end{center}
%
%    \vskip 2cm
%
%    \begin{flushright}
%	Esta é a versão original da dissertação elaborada pelo\\
%	candidato Felipe Lombardi Pierin, tal como \\
%	submetida à Comissão Julgadora.
%    \end{flushright}
%
\pagebreak

% ---------------------------------------------------------------------------- %
% Página de rosto (SÓ PARA A VERSÃO CORRIGIDA - APÓS DEFESA)
% Resolução CoPGr 5890 (20/12/2010)
%
% Nota: O título para as dissertações/teses do IME-USP devem caber em um 
% orifício de 10,7cm de largura x 6,0cm de altura que há na capa fornecida pela SPG.
%
% IMPORTANTE:
%   Coloque um '%' em todas as linhas desta
%   página antes de compilar a versáo do trabalho que será entregue
%   à Comissão Julgadora antes da defesa
%
%

\newpage
\thispagestyle{empty}
    \begin{center}
        \vspace*{2.3 cm}
        \textbf{\Large{IntegraWeb: uma proposta de arquitetura baseada em
mapeamentos semânticos e técnicas de mineração de dados}}\\
        \vspace*{2 cm}
    \end{center}

    \vskip 2cm

    \begin{flushright}
	Esta versão da dissertação contém as correções e alterações sugeridas\\
	pela Comissão Julgadora durante a defesa da versão original do trabalho,\\
	realizada em 05/12/2017. Uma cópia da versão original está disponível no\\
	Instituto de Matemática e Estatística da Universidade de São Paulo.

    \vskip 2cm

    \end{flushright}
    \vskip 4.2cm

    \begin{quote}
    \noindent Comissão Julgadora:
    
    \begin{itemize}
		\item Prof. Dr. Jaime Simão Sichman - EP-USP
		\item Profa. Dra. Renata Wassermann - IME-USP
		\item Profa. Dra. Ana Cristina Bicharra Garcia - UNIRIO
    \end{itemize}
      
    \end{quote}
\pagebreak


\pagenumbering{roman}     % começamos a numerar 

% ---------------------------------------------------------------------------- %
% Agradecimentos:
% Se o candidato não quer fazer agradecimentos, deve simplesmente eliminar esta página 
\chapter*{Agradecimentos}

Agradeço a minha família pelo apoio ao longo de todos esses anos ao longo dessa grande jornada. Nesse período tive ainda mais amor, compreensão e todo o apoio possível durante essa fase de minha vida. 

Agradeço também aos meus amigos por toda a paciência que tiveram comigo durante os momentos mais criticos desta etapa, pelo companheirismo sempre presente, pela confiança, pela alegria e pela força transmitida. Todos esses gestos me ajudaram a alcançar resultados melhores e também construíram alicerces para não desistir deste projeto. 

Agradeço ao meu amigo e colega de trabalho Daniel Deimann que dedicou parte do seu tempo para sugerir e repassar o conhecimento necessário para a construção da camada de apresentação que implementa a proposta do projeto aqui apresentado.

Agradeço aos meus professores do IME-USP e da EP-USP que mostraram pontos de vista diferentes do meu e me auxiliaram a tomar decisões mais assertivas em relação a este trabalho por meio das matérias que assisti, pela sugestão de artigos e também com a contribuição com ideias.

Agradeço ao meu orientador Jaime Sichman que acreditou no meu potencial e ofereceu uma oportunidade única para participar de um universo de conhecimento. Por sua dedicação com este projeto, pelas sugestões e por estar sempre presente.



% ---------------------------------------------------------------------------- %
% Resumo
\chapter*{Resumo}

\noindent Pierin, F. L. \textbf{IntegraWeb: uma proposta de arquitetura baseada em mapeamentos semânticos e técnicas de mineração de dados}. 
2017. 75 f.
Dissertação (Mestrado) - Instituto de Matemática e Estatística,
Universidade de São Paulo, São Paulo, 2017.
\\

%Elemento obrigatório, constituído de uma sequência de frases concisas e
%objetivas, em forma de texto.  Deve apresentar os objetivos, métodos empregados,
%resultados e conclusões.  O resumo deve ser redigido em parágrafo único, conter
%no máximo 500 palavras e ser seguido dos termos representativos do conteúdo do
%trabalho (palavras-chave). 
Atualmente uma grande quantidade de conteúdo é produzida e publicada
todos os dias na Internet. São documentos publicados por diferentes
pessoas, por diversas organizações e em inúmeros formatos sem qualquer
tipo de padronização. Por esse motivo, a informação relevante sobre um
mesmo domínio de interesse acaba espalhada pela Web nos diversos
portais, o que dificulta uma visão ampla, centralizada e objetiva
sobre esta informação. Nesse contexto, a integração dos dados
espalhados na rede torna-se um problema de pesquisa relevante, para
permitir a realização de consultas mais inteligentes, de modo a obter
resultados mais ricos de significado e mais próximos do interesse do
usuário. No entanto, tal integração não é trivial, sendo por muitas
vezes custosa devido à dependência do desenvolvimento de sistemas e
mão de obra especializados, visto que são poucos os modelos
reaproveitáveis e facilmente integráveis entre si. Assim, a existência
de um modelo padronizado para a integração dos dados e para o acesso à
informação produzida por essas diferentes entidades reduz o esforço na
construção de sistemas específicos. Neste trabalho é proposta uma
arquitetura baseada em ontologias para a integração de dados publicados na Internet. O seu uso é ilustrado através de casos de uso reais para a integração da informação na Internet, evidenciando  como o uso de ontologias pode trazer resultados mais relevantes. 
\\

\noindent \textbf{Palavras-chave:} Web semântica, integração de informação, ontologias

% ---------------------------------------------------------------------------- %
% Abstract
\chapter*{Abstract}

\noindent Pierin, F. L. \textbf{IntegraWeb: An architectural proposal based on semantic mappings and data mining techniques}. 
2017. 75 f.
Dissertação (Mestrado) - Instituto de Matemática e Estatística,
Universidade de São Paulo, São Paulo, 2017.
\\


%Elemento obrigatório, elaborado com as mesmas características do resumo em
%língua portuguesa. De acordo com o Regimento da Pós- Graduação da USP (Artigo
%99), deve ser redigido em inglês para fins de divulgação. 
A lot of content is produced and published every day on the
Internet. Those documents are published by different people,
organizations and in many formats without any type of established
standards. For this reason, relevant information about a domain of
interest is spread through the Web in various portals, which hinders a broad, centralized and objective view of this information. In this context, the integration of the data scattered in the network becomes
a relevant research problem, to enable smarter queries, in order to
obtain richer results of meaning and closer to the user's
interest. However, such integration is not trivial, and is often
costly because of the reliance on the development of specialized
systems by professionals, since there are few reusable and easily
integrable models. Thus, the existence of a standardized model for
data integration and access to the information produced by these
different entities reduces the effort in the construction of specific
systems. In this work we propose an architecture based on ontologies for the integration of data published on the Internet. Its use is illustrated through experimental cases for the integration of information on the Internet, showing how the use of ontologies can bring more relevant results.
\\

\noindent \textbf{Keywords:} semantic Web, data-integration, ontologies.

\chapter*{Preâmbulo}

Após a sessão de defesa da dissertação, os membros da banca sugeriram alterar parcialmente o título do trabalho. O novo título, mais adequado à ênfase do trabalho seria "IntegraWeb: uma proposta de arquitetura baseada em
mapeamentos semânticos". Tal alteração, entretanto, não é possível na legislação vigente da USP.

% ---------------------------------------------------------------------------- %
% Sumário
\tableofcontents    % imprime o sumário

% ---------------------------------------------------------------------------- %
\input lista_abreviaturas

% ---------------------------------------------------------------------------- %
%\chapter{Lista de Símbolos}
%\begin{tabular}{ll}
%        $\omega$    & Frequência angular\\
%        $\psi$      & Função de análise \emph{wavelet}\\
%        $\Psi$      & Transformada de Fourier de $\psi$\\
%\end{tabular}

% ---------------------------------------------------------------------------- %
% Listas de figuras e tabelas criadas automaticamente
\listoffigures            
%\listoftables            


% ---------------------------------------------------------------------------- %
% Capítulos do trabalho
\mainmatter

% cabeçalho para as páginas de todos os capítulos
\fancyhead[RE,LO]{\thesection}

\singlespacing              % espaçamento simples
%\onehalfspacing            % espaçamento um e meio

\input cap-introducao
\input cap-aspectos-basicos
\input cap-modelo
\input cap-resultados
\input cap-conclusoes        % associado ao arquivo: 'cap-conclusoes.tex'

% cabeçalho para os apêndices
\renewcommand{\chaptermark}[1]{\markboth{\MakeUppercase{\appendixname\ \thechapter}} {\MakeUppercase{#1}} }
\fancyhead[RE,LO]{}
\appendix

\chapter{Consulta SPARQL para integração de dados}
\label{ape:sparql-marco-zero}
\begin{lstlisting}[language=SPARQL,basicstyle=\ttfamily\small]
PREFIX rdf:<http://www.w3.org/1999/02/22-rdf-syntax-ns#>
PREFIX rdfs:<http://www.w3.org/2000/01/rdf-schema#>
PREFIX iweb:<http://integraweb.ddns.net/>
PREFIX schema:<http://schema.org/>
SELECT DISTINCT ?title ?latitude ?longitude ?startDate ?endDate ?startTime ?endTime ?cuisine ?description ?priceRange ?telephone ?overview ?streetAddress ?price ?type ?url ?image 
WHERE {
	?s schema:title ?titleObj ;
	schema:latitude ?latitudeObj ;
	schema:longitude ?longitudeObj ;
	rdf:type ?typeObj .
	?typeObj rdfs:subClassOf ?subClassObj .
	OPTIONAL { ?s schema:cuisine ?cuisineObj }
	OPTIONAL { ?s schema:description ?descriptionObj }
	OPTIONAL { ?s schema:endDate ?endDateObj }
	OPTIONAL { ?s schema:endTime ?endTimeObj }	
	OPTIONAL { ?s schema:overview ?overviewObj }
	OPTIONAL { ?s schema:price ?priceObj }	
	OPTIONAL { ?s schema:priceRange ?priceRangeObj }
	OPTIONAL { ?s schema:startDate ?startDateObj }
	OPTIONAL { ?s schema:startTime ?startTimeObj }
	OPTIONAL { ?s schema:streetAddress ?streetAddressObj }	
	OPTIONAL { ?s schema:telephone ?telephoneObj }
	OPTIONAL { ?s schema:serviceURL ?urlObj }
	OPTIONAL { ?s schema:image ?imageObj }
	values ?subClassObj { schema:Event schema:FoodEstablishment schema:CivicStructure } 
	BIND (str(?titleObj) as ?title)
	BIND (str(?latitudeObj) as ?latitude)
	BIND (str(?longitudeObj) as ?longitude)
	BIND (str(?cuisineObj) as ?cuisine)
	BIND (str(?descriptionObj) as ?description)
	BIND (str(?endDateObj) as ?endDate)
	BIND (str(?overviewObj) as ?overview)
	BIND (str(?priceObj) as ?price)
	BIND (str(?priceRangeObj) as ?priceRange)
	BIND (str(?startDateObj) as ?startDate)
	BIND (str(?streetAddressObj) as ?streetAddress)
	BIND (str(?telephoneObj) as ?telephone)
	BIND (str(?urlObj) as ?url)
	BIND (str(?imageObj) as ?image)
	BIND ( strafter(strafter( str(?typeObj), "http://" ),"/") as ?type )
	BIND ( strafter( str(?endTimeObj), "T" ) as ?endTime )
	BIND ( strafter( str(?startTimeObj), "T" ) as ?startTime )
	FILTER (?latitude > '-23.539606783940812' && ?latitude < '-23.55759321605919')
	FILTER (?longitude > '-46.62938980581791' && ?longitude < '-46.6490101941821')
	FILTER NOT EXISTS {
		?s rdf:type ?subtype .
		?subtype rdfs:subClassOf* ?typeObj .
		filter ( ?subtype != ?typeObj )
	}
 }
LIMIT 100
\end{lstlisting}
\chapter{Consulta com várias restrições}
\label{ape:consultas-varias-restricoes}
\begin{lstlisting}[language=SPARQL,basicstyle=\ttfamily\small]
PREFIX rdf:<http://www.w3.org/1999/02/22-rdf-syntax-ns#>
PREFIX rdfs:<http://www.w3.org/2000/01/rdf-schema#>
PREFIX iweb:<http://integraweb.ddns.net/>
PREFIX schema:<http://schema.org/>
SELECT ?title ?latitude ?longitude ?startDate ?endDate ?startTime ?endTime ?cuisine ?description ?priceRange ?telephone ?overview ?streetAddress ?price ?type ?url ?image ?servesCuisine WHERE {
	{
		?s schema:title ?titleObj ;
		schema:latitude ?latitudeObj ;
		schema:longitude ?longitudeObj ;
		iweb:near ?nearObj ; 
		rdf:type schema:Restaurant ;
		schema:servesCuisine ?servesCuisineObj ;
		rdf:type ?typeObj .
		?nearObj schema:title ?nearObjTitle .
		OPTIONAL { ?s schema:cuisine ?cuisineObj }
		OPTIONAL { ?s schema:description ?descriptionObj }
		OPTIONAL { ?s schema:endDate ?endDateObj }
		OPTIONAL { ?s schema:endTime ?endTimeObj }	
		OPTIONAL { ?s schema:overview ?overviewObj }
		OPTIONAL { ?s schema:price ?priceObj }	
		OPTIONAL { ?s schema:priceRange ?priceRangeObj }
		OPTIONAL { ?s schema:startDate ?startDateObj }
		OPTIONAL { ?s schema:startTime ?startTimeObj }
		OPTIONAL { ?s schema:streetAddress ?streetAddressObj }	
		OPTIONAL { ?s schema:telephone ?telephoneObj }
		OPTIONAL { ?s schema:serviceURL ?urlObj }
		OPTIONAL { ?s schema:image ?imageObj }
		values ?subClassObj { schema:Event schema:FoodEstablishment schema:CivicStructure } 
		BIND (str(?titleObj) as ?title)
		BIND (str(?latitudeObj) as ?latitude)
		BIND (str(?longitudeObj) as ?longitude)
		BIND (str(?servesCuisineObj) as ?servesCuisine)
		BIND (str(?descriptionObj) as ?description)
		BIND (str(?endDateObj) as ?endDate)
		BIND (str(?overviewObj) as ?overview)
		BIND (str(?priceObj) as ?price)
		BIND (str(?priceRangeObj) as ?priceRange)
		BIND (str(?startDateObj) as ?startDate)
		BIND (str(?streetAddressObj) as ?streetAddress)
		BIND (str(?telephoneObj) as ?telephone)
		BIND (str(?urlObj) as ?url)
		BIND (str(?imageObj) as ?image)
		BIND ( strafter(strafter( str(?typeObj), "http://" ),"/") as ?type )
		BIND ( strafter( str(?endTimeObj), "T" ) as ?endTime )
		BIND ( strafter( str(?startTimeObj), "T" ) as ?startTime )
		FILTER (str(?nearObjTitle) = 'Fiaminghi - Pensamentos Compostos')
		FILTER (?servesCuisine = 'Natural')
		FILTER (?type = 'Restaurant')
		FILTER NOT EXISTS {
			?s rdf:type ?subtype .
			?subtype rdfs:subClassOf* ?typeObj .
			filter ( ?subtype != ?typeObj )
		}
	}
	UNION 
	{
		?s schema:title ?titleObj ;
		schema:latitude ?latitudeObj ;
		schema:longitude ?longitudeObj ;
		rdf:type schema:ExhibitionEvent ;
		rdf:type ?typeObj .
		OPTIONAL { ?s schema:servesCuisine ?servesCuisineObj }
		OPTIONAL { ?s schema:description ?descriptionObj }
		OPTIONAL { ?s schema:endDate ?endDateObj }
		OPTIONAL { ?s schema:endTime ?endTimeObj }	
		OPTIONAL { ?s schema:overview ?overviewObj }
		OPTIONAL { ?s schema:price ?priceObj }	
		OPTIONAL { ?s schema:priceRange ?priceRangeObj }
		OPTIONAL { ?s schema:startDate ?startDateObj }
		OPTIONAL { ?s schema:startTime ?startTimeObj }
		OPTIONAL { ?s schema:streetAddress ?streetAddressObj }	
		OPTIONAL { ?s schema:telephone ?telephoneObj }
		OPTIONAL { ?s schema:serviceURL ?urlObj }
		OPTIONAL { ?s schema:image ?imageObj }
		values ?subClassObj { schema:Event schema:FoodEstablishment schema:CivicStructure } 
		BIND (str(?titleObj) as ?title)
		BIND (str(?latitudeObj) as ?latitude)
		BIND (str(?longitudeObj) as ?longitude)
		BIND (str(?servesCuisineObj) as ?servesCuisine)
		BIND (str(?cuisineObj) as ?cuisine)
		BIND (str(?descriptionObj) as ?description)
		BIND (str(?endDateObj) as ?endDate)
		BIND (str(?overviewObj) as ?overview)
		BIND (str(?priceObj) as ?price)
		BIND (str(?priceRangeObj) as ?priceRange)
		BIND (str(?startDateObj) as ?startDate)
		BIND (str(?streetAddressObj) as ?streetAddress)
		BIND (str(?telephoneObj) as ?telephone)
		BIND (str(?urlObj) as ?url)
		BIND (str(?imageObj) as ?image)
		BIND ( strafter(strafter( str(?typeObj), "http://" ),"/") as ?type )
		BIND ( strafter( str(?endTimeObj), "T" ) as ?endTime )
		BIND ( strafter( str(?startTimeObj), "T" ) as ?startTime )
		FILTER (?title = 'Fiaminghi - Pensamentos Compostos')
		FILTER (?type = 'ExhibitionEvent')
		FILTER NOT EXISTS {
			?s rdf:type ?subtype .
			?subtype rdfs:subClassOf* ?typeObj .
			filter ( ?subtype != ?typeObj )
		}
	}
	UNION 
	{
		?s schema:title ?titleObj ;
		schema:latitude ?latitudeObj ;
		schema:longitude ?longitudeObj ;
		iweb:near ?nearObj ; 
		rdf:type schema:BarOrPub ;
		rdf:type ?typeObj .
		?nearObj schema:title ?nearObjTitle .
		OPTIONAL { ?s schema:servesCuisine ?servesCuisineObj }
		OPTIONAL { ?s schema:description ?descriptionObj }
		OPTIONAL { ?s schema:endDate ?endDateObj }
		OPTIONAL { ?s schema:endTime ?endTimeObj }	
		OPTIONAL { ?s schema:overview ?overviewObj }
		OPTIONAL { ?s schema:price ?priceObj }	
		OPTIONAL { ?s schema:priceRange ?priceRangeObj }
		OPTIONAL { ?s schema:startDate ?startDateObj }
		OPTIONAL { ?s schema:startTime ?startTimeObj }
		OPTIONAL { ?s schema:streetAddress ?streetAddressObj }	
		OPTIONAL { ?s schema:telephone ?telephoneObj }
		OPTIONAL { ?s schema:serviceURL ?urlObj }
		OPTIONAL { ?s schema:image ?imageObj }
		values ?subClassObj { schema:Event schema:FoodEstablishment schema:CivicStructure } 
		BIND (str(?titleObj) as ?title)
		BIND (str(?latitudeObj) as ?latitude)
		BIND (str(?longitudeObj) as ?longitude)
		BIND (str(?servesCuisineObj) as ?servesCuisine)
		BIND (str(?cuisineObj) as ?cuisine)
		BIND (str(?descriptionObj) as ?description)
		BIND (str(?endDateObj) as ?endDate)
		BIND (str(?overviewObj) as ?overview)
		BIND (str(?priceObj) as ?price)
		BIND (str(?priceRangeObj) as ?priceRange)
		BIND (str(?startDateObj) as ?startDate)
		BIND (str(?streetAddressObj) as ?streetAddress)
		BIND (str(?telephoneObj) as ?telephone)
		BIND (str(?urlObj) as ?url)
		BIND (str(?imageObj) as ?image)
		BIND ( strafter(strafter( str(?typeObj), "http://" ),"/") as ?type )
		BIND ( strafter( str(?endTimeObj), "T" ) as ?endTime )
		BIND ( strafter( str(?startTimeObj), "T" ) as ?startTime )
		FILTER (str(?nearObjTitle) = 'Fiaminghi - Pensamentos Compostos')
		FILTER (?type = 'BarOrPub')
		FILTER NOT EXISTS {
			?s rdf:type ?subtype .
			?subtype rdfs:subClassOf* ?typeObj .
			filter ( ?subtype != ?typeObj )
		}
	}
}
\end{lstlisting}

%\include{ape-conjuntos}      % associado ao arquivo: 'ape-conjuntos.tex'

% ---------------------------------------------------------------------------- %
% Bibliografia
\backmatter \singlespacing   % espaçamento simples
%\bibliographystyle{alpha-ime} % citação bibliográfica textual
\bibliographystyle{plainnat-ime} % citação bibliográfica textual
\bibliography{bibliografia}  % associado ao arquivo: 'bibliografia.bib'

% ---------------------------------------------------------------------------- %
% índice remissivo
%\index{TBP|see{periodicidade região codificante}}
%\index{DSP|see{processamento digital de sinais}}
%\index{STFT|see{transformada de Fourier de tempo reduzido}}
%\index{DFT|see{transformada discreta de Fourier}}
%\index{Fourier!transformada|see{transformada de Fourier}}

\printindex   % imprime o índice remissivo no documento 

\end{document}