%% ------------------------------------------------------------------------- %%
\chapter{Conclusões e Trabalhos Futuros}
\label{cap:conclusoes}

A Web Semântica tem o potencial de conectar os dados espalhados nas mais diferentes fontes de informação na Web, contidos em diferentes portais e representados em diferentes formatos. A premissa deste trabalho é que ao combinar a tecnologia de Web Semântica com técnicas de mineração e integração de dados na Web, permite-se que o conteúdo relevante dos portais espalhados pela Internet possa ser extraído de maneira automatizada para oferecer um resultado muito mais expressivo ao usuário final.

A aplicação de ontologias, auxilia a elevar a qualidade da informação exposta na Internet. Através de seu uso, é possível atribuir significado à informação presente na Internet, abrindo assim um leque de possibilidades ao propiciar consultas mais ricas, com semântica agregada, proveniente de fontes e formatos de dados diversas, com informações mais próximas dos interesses dos indivíduos, auxiliando-os na tomada de decisões do dia-a-dia.

Este trabalho dá um primeiro passo nesta direção, com uma proposta de arquitetura capaz de permitir a integração da informação provenientes de portais distintos a partir do mapeamento para ontologias bem estabelecidas e a centralização da informação para consulta aos dados com o uso de ontologias. 

%------------------------------------------------------
\section{Conclusões} 
\label{sec:consideracaoes_finais}

Há grandes obstáculos a serem superados no que diz respeito à extração de dados. Como a maioria dos portais não anota semanticamente o seu conteúdo que é gerado dinamicamente, torna-se necessária a aplicação de artifícios para a recuperação de conteúdo a partir da estrutura sintática desses documentos.  Propostas como a apresentada neste trabalho são vulneráveis à mudança da maneira pela qual a informação é exposta para os usuários. Em outras palavras, se um determinado portal muda sua forma de apresentação para os seus usuários, o processo  de extração de dados deve ser atualizado. Além disso, quanto maior a frequência de modificação da estrutura desses documentos, maior é a quantidade de manutenção na infra-estrutura de recuperação do conteúdo do portal.

A própria natureza da Internet é um obstáculo para a recuperação de dados quando há a necessidade de se percorrer variados portais da Web para a construção de um resultado mais abrangente. Isso porque a navegação link a link, página a página, depende de que o servidor alvo tenha capacidade de expor todo o conteúdo necessário para a consulta de maneira rápida e sem interrupções. No entanto, esse tipo de solução fica a mercê da velocidade do servidor, da disponibilidade da informação, da estrutura de rede ou até mesmo de detalhes como a cota de tráfego de dados de um portal ser excedida. Além disso, é inviável a recuperação de todo o conteúdo de grandes portais a cada consulta, uma vez que essas organizações possuem bases gigantescas de dados. A centralização de conteúdo em uma base de dados semântica  diminui os problemas intrínsecos à rede e permite que o conhecimento possa ser construído e atualizado de maneira mais gradual, o que tende a oferecer mais informações a qualquer momento e independe do gargalo ocasionado pela consulta federada em várias fontes de informação de maneira simultânea. Ainda nessa linha, ao centralizar a informação em uma base de dados única, o processo de inferência sobre dados não precisa ser feito sob demanda a cada consulta realizada por um usuário, mas de maneira assíncrona, o que implica em resultados mais rápidos.

A integração da informação de dados não necessita ser confinada a uma definição de conceito especifica e suas derivações. Neste trabalho, demonstrou-se que diferentes domínios da informação podem ser avaliados de maneira conjunta: foram tratados eventos como exposições, peças de teatro e shows, bem como restaurantes, bares ou estações de metrô. 

A resolução de conflitos de informação acontece quando um determinado dado possui referências em portais distintos na Internet mas com informações diferentes expostas. Há várias formas de tratar esse problema e a seleção de uma depende, entre outros fatores, do domínio do conteúdo escolhido. No contexto da extração de conhecimento sobre eventos, este trabalho demonstrou que a priorização da fonte dos dados é um mecanismo que pode ser utilizado para decidir qual informação mostrar para o usuário final. Além disso, quando as informações dizem respeito a um mesmo conteúdo, torna-se possível agregá-las de maneira a oferecer uma informação mais completa e consequentemente mais relevante, conforme os experimento visto na seção~\ref{sec:resolucao_de_conflitos_mesmo_evento}.

Neste trabalho, diferentemente de projetos como o VirtuosoRDF e o D2RQ que se valem das definições estruturais de tabelas, colunas entre outras características bem definidas no banco de banco para promover a junção das fontes de informação através do mapeamento semântico para RDF, o foco é o conteúdo publicado na Internet, conteúdo este que muitas vezes não está disponível para o usuário senão por meio de uma camada de apresentação que faz o uso de documentos com padrões bastante distintos. Além disso, o processo de anotação do conteúdo de um portal é feito de forma manual uma vez que o foco da proposta apresentada é mesclar a informação espalhada na Web e não a facilitação de marcação semântica assim como é proposta nos trabalhos de \citep{AndreasHess} e no DeepAnnotation. Finalmente, no que diz respeito a extração de informação em documentos estruturados, este trabalho se assemelha ao de \citep{May} para a integração de documentos distintos mas diferente na forma pela qual a integração é realizada uma vez que na proposta aqui apresentada a informação é recuperada por meio do uso de expressões regulares e então transformada em RDF.

%------------------------------------------------------

Espera-se que este trabalho possa ser mais um incentivo para o reuso de informação exposta na Internet e para o avanço da Web Semântica. Trata-se de uma amostra de que a nova proposta da Web na qual todos os dados estão interconectados não é uma utopia e pode estar mais próxima do que imaginamos. 

%------------------------------------------------------
\section{Possíveis Extensões} 
\label{sec:sugestoes_para_persquisas_futuras}

Almeja-se que a proposta apresentada neste trabalho possa ser tomada como referência para a construção de sistemas mais inteligentes, capazes de unificar sob um ponto de vista mais amplo os mais diferentes domínios da informação para, dessa maneira, oferecer conteúdos mais ricos para os futuros usuários da Internet. São inúmeros os domínios que poderiam ser abordados pela proposta aqui apresentada, entre eles, uma possível aplicação seria o de, por exemplo, facilitar a busca de hotéis e pousadas próximos a aeroportos. 
 
Atualmente a busca de passagens aéreas em diferentes portais não integra a informação da localização dos aeroportos e horário de partida e chegada dos diferentes voos com a informação sobre hotéis e pousadas abertos no dia e no horário da chegada da viagem. Dessa maneira, ao simular uma data de chegada em uma determinada cidade, esses portais indicam outros portais agregadores de conteúdo com as informações sobre hotéis na região. Acontece que muitas vezes esses hotéis estão fechados, sem disponibilidade de vaga, fora de uma faixa de valor de diária aceitável, possivelmente sem serviços adicionais que possam ser de interesse de um individuo e até mesmo muito longe do aeroporto do desembarque. Nesse contexto, aplicando a arquitetura proposta neste trabalho e utilizando ontologias que definem o domínio Hotel e Voo como, por exemplo, os conceitos Flight\footnote{\url{http      ://schema.org/Flight}} e Hotel\footnote{\url{http://schema.org/Hotel}} descritos no portal Schema.org, tal pesquisa poderia ser simplificada pois a partir de uma única consulta tornaria-se possível filtrar todas as opções disponíveis nos diversos portais agregadores de informações dentro do assunto.

A conclusão desse trabalho abre portas para novas pesquisas voltadas para a melhoria da integração da informação na Web. Algumas áreas abordadas na proposta de arquitetura apresentada neste documento ainda podem ser bastante exploradas a fim de torná-la ainda mais eficiente. Entre as sugestões futuras, encontram-se: (i)  a pesquisa de mecanismos mais eficientes de reconhecimento e recuperação de informação relevantes, (ii) a auto-detecção e mapeamento em tempo real dos dados contidos em documentos Web para documentos semanticamente anotados, (iii) a construção de motores capazes de converter consultas SPARQL em HTML, assim como hoje algumas consultas SPARQL podem ser traduzidas diretamente para SQL e (iv)  mecanismos de extração de informação capazes de reconhecer a mudança da estrutura sintática das páginas HTML e que tenham a habilidade de se auto ajustar a elas, sem a necessidade de intervenção humana, por exemplo através de técnicas de aprendizado de máquina.